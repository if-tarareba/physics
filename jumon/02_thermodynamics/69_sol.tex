%!TEX root = *.tex
%%%%%%%%%%%%%%%%%%
% メモ
\setlength{\baselineskip}{18pt}
\begin{comment}

\end{comment}
% カウンタのリセット
\setcounter{figure}{0}
% 解答
\qSentence{【解答】}

\hang{(1)}
内部エネルギーの変化を定積モル比熱を用いて表すと,
\begin{align*}
  &\varDelta U_{\A\B} = \underline{nC_v(T_\B-T_\A)},
  &\varDelta U_{\B\C} = \underline{nC_v(T_\C-T_\B)}, \\
  &\varDelta U_{\C\D} = \underline{nC_v(T_\D-T_\C)},
  &\varDelta U_{\D\A} = \underline{nC_v(T_\A-T_\D)}.
\end{align*}

\hang{(2)}
A$\rightarrow$ B,C$\rightarrow$ Dは断熱変化なので,$Q_{\A\B}=\underline{0},\,Q_{\C\D}=\underline{0}$.
B$\rightarrow$ Cは定圧変化なので,\mbox{$Q_{\B\C} = \underline{nC_p(T_\C-T_\B)}$}.
D$\rightarrow$ Aは定積変化なので,\mbox{$Q_{\D\A} = \underline{nC_v(T_\A-T_\D)}$}.

\hang{(3)}
D$\rightarrow$Aは定積変化ゆえ,$W_{\D\A}=\underline{0}$.
また,熱力学第一法則より
\begin{align*}
  W_{\A\B} &= Q_{\A\B} - \varDelta U_{\A\B} = \underline{-nC_v(T_\B-T_\A)}, \\
  W_{\B\C} &= Q_{\B\C} - \varDelta U_{\B\C} = \underline{n(C_p-C_v)(T_\C-T_\B)}, \\
  W_{\C\D} &= Q_{\C\D} - \varDelta U_{\C\D} = \underline{-nC_v(T_\D-T_\C)}.
\end{align*}

\hang{(4)}
A$\rightarrow$ B,C$\rightarrow$ Dが断熱変化なので$T_\D>T_\A,\,T_\C>T_\B$である.
この熱サイクルにおいて吸熱する過程,すなわち(2)の解$Q$が正の値をとるのはB$\rightarrow$ Cのみである.
一方で,気体がする正味の仕事$W$は
\begin{align*}
  W &= W_{\A\B} + W_{\B\C} + W_{\C\D} + W_{\D\A} \\
  &= Q_{\B\C} - (\varDelta U_{\A\B} + \varDelta U_{\B\C} + \varDelta U_{\C\D}) \qquad (\because \text{熱力学第一法則})\\
  &= Q_{\B\C} - nC_v(T_\D-T_\A) \\
\end{align*}
以上より熱効率$e$は,$\gamma =\tfrac{C_p}{C_v}$を用いて
\begin{align*}
  e &= \dfrac{W}{Q_{\B\C}} = \dfrac{Q_{\B\C} - nC_v(T_\D-T_\A)}{Q_{\B\C}} \\
  &= 1- \dfrac{C_v}{C_p}\cdotp\dfrac{T_\D-T_\A}{T_\C-T_\B} \\
  &= \underline{1-\dfrac{T_\D-T_\A}{\gamma (T_\C-T_\B)}}
\end{align*}

%%%%%%%%%%%%%%%%%%
