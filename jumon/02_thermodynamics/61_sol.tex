%!TEX root = *.tex
%%%%%%%%%%%%%%%%%%
% メモ
\begin{comment}

\end{comment}
% カウンタのリセット
\qSentence{【解答】}
% 解答
\hang{(1)}
熱量について,$Q=mC_w T_B\,\unit{J}$.

\hang{(2)}
水の温度を0℃から$T_B\unit{℃}$まで上昇させるのに
一定電力$P\unit{W}$で時間$(t_B-t_2)\unit{s}$だけ加熱したので
\begin{align*}
  P(t_B-t_2) &= mC_wT_B\\
  \therefore\, P &= \frac{mC_wT_B}{t_B-t_2}\ \unit{W}
\end{align*}

\hang{(3)}
氷の融解熱を$q_m\unit{J/g}$とする.融解には時間が$t_2-t_1\unit{s}$だけかかったので
\begin{align*}
  mq_m &= P(t_2-t_1)\\
  \therefore\, q_m &= \frac{t_2-t_1}{t_B-t_2}\,C_w T_B\ \unit{J/g}
\end{align*}

\hang{(4)}
氷の比熱を$C_i$〔J/(g$\cdot$K)〕とすると,
氷が$-T_1\unit{℃}$から0℃までに必要な熱量は$mC_iT_1\unit{℃}$であり,
電力$P\unit{W}$で時間$t_1\unit{s}$の間に与えた熱量なので
\begin{align*}
  mC_iT_1 &= Pt_1\\
  mC_iT_1 &= \frac{mC_wT_B}{t_B-t_2}t_1\\
  \therefore\, \frac{C_i}{C_w} &= \frac{T_B}{T_1}\frac{t_1}{t_B-t_2}
\end{align*}

\hang{(5)}
時刻$t_A\unit{s}$において,すでに融けて水になっている部分(質量$m_w$)は,
$(t_A-t_1)\unit{s}$の間に融けており,
まだ融けていない部分(質量$m_i$)は,$(t_2-t_A)\unit{s}$の間に融ける.
一定電力で熱しているので,融解の過程において単位時間あたりに融ける氷の質量は一定である.
水と氷の質量の比は時間の比に等しく,
\begin{align*}
  \frac{m_i}{m_w} &= \frac{t_2-t_A}{t_A-t_1}\\
\end{align*}

\hang{(6)}
蒸発熱を$q_v\unit{J/g}$とする.
蒸発に要した熱量は$P(t_4-t_3)\unit{J}$より$mq_v=P(t_4-t_3)$が成り立つ.
また,$mq_m = P(t_2-t_1)$より
\begin{align*}
  \frac{q_v}{q_m} &= \frac{t_4-t_3}{t_2-t_1}
\end{align*}

%%%%%%%%%%%%%%%%%%
