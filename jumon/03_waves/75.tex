%!TEX root = *.tex
%%%%%%%%%%%%%%%%%%
% カウンタのリセット
% 問題文
ある媒質中を\x 軸の正の向きに速さ$v$で減衰することなく進行している連続波を考える.
この波の振幅を$A$,周期を$T$とすると,
\x 軸上の原点Oでの媒質の変位は時刻$t$の関数として$y=A\sin\dfrac{2\pi}{T}t$で表される.
これを入射波として$x=L\,(L>0)$の位置で固定端反射させる.
反射による波の減衰は無視できるとする.

\begin{enumerate}[(1)]
  \setlength{\leftskip}{-1.5zw}
  \setlength{\itemindent}{1zw}\setlength{\labelsep}{0.5zw}
  \setlength{\labelwidth}{1zw}\setlength{\leftmargin}{1zw}
  \setlength{\itemsep}{0.5\baselineskip}
  \item 入射波の振動数$f$と波長$\lambda$を$v$と$T$で表せ.
  \item $x<L$における入射波を,$v$と$T$を用いて$t$の関数として表せ.
  \item (2)の結果を用いて,反射波を$x$および$t$の関数として表せ.
  \item 入射波と反射波が重なりあって波形の進行しないように観測される波,つまり定在波(定常波)ができることを,式を使って説明せよ.
  \item $\lambda=\dfrac{4}{5}L$の場合について,(4)の定在波が最大振幅になるときの波形の概略を描け.
\end{enumerate}



% メモ
\begin{comment}

\end{comment}


%%%%%%%%%%%%%%%%%%
