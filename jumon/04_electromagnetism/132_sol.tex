%!TEX root = *.tex
%%%%%%%%%%%%%%%%%%
% メモ
\begin{comment}

\end{comment}
% カウンタのリセット
% 解答
\noindent {\large【解答】\par}
\noindent (1)\,
向きは上向き.
また,コイルAは1mあたりの巻数\nn より磁束密度は$\mu nI_A$であり,
\begin{align*}
  \varPhi  = \mu nI_A S\unit{Wb}
\end{align*}

\noindent (2)\,
レンツの法則よりコイルBに生じる誘導起電力は正である.
ファラデーの法則より
\begin{align*}
  V_B = N\dfrac{\Delta\varPhi}{\Delta t}\unit{V}
\end{align*}

\noindent (3)\,
(1)の結果より,$\Delta\varPhi = \mu n\Delta I_A S$.したがって
\begin{align*}
  V_B = \mu nNS \frac{\Delta I_A}{\Delta t}\unit{V}
\end{align*}

\noindent (4)\,
(3)の結果から$M=\mu nNS\unit{H}$

\noindent (5)\,
レンツの法則より誘導起電力は上向きに生じる.
オームの法則より,
\begin{align*}
  -\mu nNS\frac{\Delta I_A}{\Delta t} &= RI_B\\
  \therefore I_B &= -\frac{\mu nNS}{R}\frac{\Delta I_A}{\Delta t}
\end{align*}


%%%%%%%%%%%%%%%%%%
