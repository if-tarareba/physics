%!TEX root = *.tex
%%%%%%%%%%%%%%%%%%
% メモ
\begin{comment}

\end{comment}
% カウンタのリセット
% 解答
\noindent {\large【解答】\par}
\noindent (1)\,
\x 成分:$v\sin\theta\unit{m/s}$,
\z 成分:$v\cos\theta\unit{m/s}$.

\noindent (2)\,
ローレンツ力の大きさ$F$は,
\begin{align*}
  F &= e\cdot (v\sin\theta)\cdot B \\
  &= evB\sin\theta\unit{N}
\end{align*}

\noindent (3)\,
ローレンツ力を向心力とした等速円運動を行うので,
\begin{align*}
  m\frac{(v\sin\theta)^2}{R} &= evB\sin\theta \\
  R &= \frac{mv\sin\theta}{eB}\unit{m}
\end{align*}

\noindent (4)\,
周期$T$は,
\begin{align*}
  T &= \frac{v\sin\theta}{2\pi R} = \frac{v\sin\theta}{2\pi} \frac{eB}{mv\sin\theta} \\
  &= \frac{eB}{2\pi m}
\end{align*}

\noindent (5)\,
周期$T$の間に\z 軸方向に進む距離がピッチ$l\unit{m}$であるので,
\begin{align*}
  l = v\cos\theta T = \frac{eBv\cos\theta}{2\pi m}\unit{m}
\end{align*}

\noindent (6)\,
電位差で加速する間に電子が受けるエネルギーが$eE\unit{J}$であり,
これが電子の運動エネルギー$\dfrac{1}{2}mv^2\unit{J}$となる.
\begin{align*}
  \frac{1}{2}mv^2 &= eE 
  \intertext{$v$は速さゆえ正なので}
  v &= \sqrt{\frac{2eE}{m}}\unit{m/s}
\end{align*}

\noindent (7)\,
(3),(6)より
\begin{align*}
  R &= \frac{mv\sin\theta}{eB} = \frac{\sin\theta}{B} \sqrt{\frac{2mE}{e}}
  \intertext{両辺を二乗して}
  R^2 &= \left(\frac{\sin\theta}{B}\right)^2 \sqrt{\frac{2mE}{e}} \\
  \frac{e}{m} &= 2E\left(\frac{\sin\theta}{BR}\right)^2 \unit{C/kg}
\end{align*}


%%%%%%%%%%%%%%%%%%
