%!TEX root = *.tex
%%%%%%%%%%%%%%%%%%
% メモ
\begin{comment}

\end{comment}
% カウンタのリセット
\setcounter{eqNo}{0}
% 解答
\noindent {\large【解答】\par}

\noindent 〔A〕\par 
\noindent \,(1)\,
$T$を糸の張力とし,$a$をおもりAが下降する加速する加速度の大きさとする.
おもりA,Bについて運動方程式
\begin{align*}
  m_1 a = m_1 g -T \assignEqNo\\
  m_2 a = T - m_2 g \assignEqNo
\end{align*}
\ctext{1},\ctext{2}より
\begin{align*}
  (m_1 + m_2)a &= (m_1 - m_2)g\\
  a &= \dfrac{m_1-m_2}{m_1+m_2}g
\end{align*}

\noindent \,(2)\,
\ctext{1}と(1)の結果から,
\begin{align*}
  T = \dfrac{2m_1m_2}{n_1+m_2}g
\end{align*}
である.

\noindent \,(3)\,
$m_1=m_2=m$(定数)とおく.このとき,(2)の結果から,
\begin{align*}
  T = \dfrac{2m_1(m-m_1)}{m_1+m_2}g
\end{align*}
となる.これより,$T$は$m_1=\dfrac{m}{2}$のとき最大である.すなわち,
$m_1=m_2$.

\noindent 〔B〕\par 
\noindent\,(4)
$C$が糸からうける張力を$T_1$,Dが糸からうける張力を$T_2$とする.
また,$\alpha,\,\beta,\,\gamma$をそれぞれ,C,D,Eの加速度とし,すべて鉛直下向きを正とする.
このとき,運動方程式
\begin{align*}
  3M \alpha &= 3Mg - T_1 \assignEqNo\\
  2M \beta &= 2Mg - T_2 \assignEqNo\\
  M \gamma &= Mg - T_3 \assignEqNo
\end{align*}
加えて動滑車は糸を介しておもりCとつながっているので,加速度は$-\alpha$であり,運動方程式より
\begin{align*}
  0 = 2T_2 - T_1 \assignEqNo
\end{align*}
また,DとEも動滑車を間にして糸でつながっているので,動滑車に対する相対加速度の和はゼロベクトルとなるので,
\begin{align*}
  (\beta - (-\alpha)) + (\gamma - (-\alpha)) &= 0 \\
  \therefore 2\alpha + \beta + \gamma &= 0
\intertext{となり,両辺に$M$を乗じて}
  2M\alpha + M\beta + M\gamma &= 0 \assignEqNo
\end{align*}
を得る.
いま,\ctext{3},\ctext{4},\ctext{5}を\ctext{7}に代入して
\begin{align*}
  &2(Mg-\dfrac{1}{3}T_1) + (Mg - \dfrac{1}{2}T_2) + (Mg-T_2) = 0\\
  &\ \therefore 4T_1 + 9T_2 = 24Mg \assignEqNo
\end{align*}
\ctext{6},\ctext{8}を連立して解くと
\begin{align*}
  T_1 = \dfrac{48}{17}Mg,\quad T_2 = \dfrac{24}{17}Mg
\end{align*}
であり,\ctext{3},\ctext{4},\ctext{5}より
\begin{align*}
  \alpha = \dfrac{1}{17}g,\ \beta = \dfrac{5}{17}g,\ \gamma = -\dfrac{7}{17}g
\end{align*}
以上より,C,D,Eの加速度の大きさはそれぞれ
$\dfrac{1}{17}g,\,\dfrac{5}{17}g,\,\dfrac{7}{17}g$.

\noindent \,(5)\, (4)の過程より$T_2=\dfrac{24}{17}Mg$.

\noindent \,(6)\, Cが静止しているということは,動滑車も静止している.
すなわち,定滑車とみなすことができる.
よって,D,Eの運動は〔A〕で$m_1=2M,\,m_2=M$としたときの運動に一致する.
したがってDがうける張力の大きさ$S_2$は
\begin{align*}
  S_2 = \dfrac{2m_1m_2}{m_1+m_2}g = \dfrac{4}{3}Mg
\end{align*}
さらに,動滑車にはたらく力のつりあいより,$\C^\prime$がうける張力の大きさは$\dfrac{8}{3}Mg$である.よって,$\C^\prime$の運動方程式より
\begin{align*}
  M^\prime\cdot 0 &= M^\prime g - \dfrac{8}{3}Mg \\
  \therefore M^\prime &= \dfrac{8}{3}M 
\end{align*}

%%%%%%%%%%%%%%%%%%
