%!TEX root = *.tex
%%%%%%%%%%%%%%%%%%
% メモ
\begin{comment}

\end{comment}
% カウンタのリセット
% 解答
\noindent {\large【解答】\par}

\noindent (1)\,
弦は基本振動をするので波長$\lambda_0 = 2l$.

\noindent (2)\,
弦を伝わる波の周波数と音波の周波数は一致して,$f=\frac{V}{\lambda_0}=\frac{V}{2l}$.
Bのときに基本振動,Cのときに3倍振動をするということは,
BC間の距離が音波の半波長となるので,音波の波長$\lambda =2(d_2-d_1)$.
よって,音波の速さ$v=f\lambda =\frac{d_2-d_1}{l}V$.

また,開口端補正とは,音波の波長の4分の1とABの距離の差のことであるから
\begin{align*}
  \Delta x &= \frac{\lambda}{4}-d_1 = \frac{d_2-d_1}{2}-d_1 \\
  &= \frac{d_2-3d_1}{2}
\end{align*}


\noindent (3)\,
Cからさらに半波長($d_2-d_1$)だけ下がった位置で5倍振動が起こる.すなわち,\mbox{$d_2+(d_2-d_1)=2d_2-d_1$.}

\noindent (4)\,
開菅にしたときに起こる共鳴は3倍振動である.管の両端に開口端補正があることに注意して,菅の全長$L$は
\begin{align*}
  L+2\Delta x = \frac{3}{2}\lambda \quad
  \therefore L = 2d_2
\end{align*}

\noindent (5)\,
弦の長さ$l$が短くなると,弦を伝わる周波数は大きくなる.
音速一定ゆえ,音波の波長は短くなる.
よって,$l^\prime$のときの気柱のおける振動は4倍振動である.

弦を伝わる波長${\lambda_0}^\prime = 2l^\prime$なので,
周波数$f^\prime = \frac{V}{2l^\prime}$.
よって音波の波長は\mbox{$\lambda^\prime = \frac{v}{f^\prime}=\frac{2l^\prime}{l}(d_2-d_1)=\frac{l^\prime}{l}\lambda$}.

開口端補正は一定なので,管の全長について,
\begin{align*}
  \frac{3}{2}\lambda &= 2\lambda^\prime \\
  \therefore l^\prime &= \frac{3}{4}l
\end{align*}

%%%%%%%%%%%%%%%%%%%
