%!TEX root = *.tex
%%%%%%%%%%%%%%%%%%
% メモ
\begin{comment}

\end{comment}
% カウンタのリセット
% 解答
\qSentence{【解答】}

\hang{I\,(1)}
完全弾性衝突なので相対速度が変わらない.
\begin{align*}
  u_0 = -(u_1+v_1)
\end{align*}

\hang{\ \,(2)}
エネルギー保存則より
\begin{align*}
  \dfrac{1}{2}M{v_1}^2 &= \dfrac{1}{2}kx^2 \qquad
  \therefore\ x = v_1\sqrt{\dfrac{M}{k}}
\end{align*}

\hang{\ \,(3)}
ばねの単振動の半周期分経ってはじめて自然長に戻るので
\begin{align*}
  T = \frac{1}{2}\cdotp 2\pi\sqrt{\frac{M}{k}} = \pi\sqrt{\frac{M}{k}}
\end{align*}

\hang{II\,(1)}
運動量保存則より
\begin{align*}
  Mv_1 &= (M+2M)v_2 \qquad
  \therefore\ v_2 = \frac{v_1}{3}
\end{align*}

\hang{\quad(2)}
エネルギー保存則より
\begin{align*}
  \frac{1}{2}M{v_1}^2 &= \frac{1}{2}(M+2M){v_2}^2 + \frac{1}{2}ky^2 \\
  ky^2 &= \frac{2}{3}M{v_1}^2 \\
  \therefore\ y &= v_1\sqrt{\frac{2M}{3k}}
\end{align*}

\hang{\quad(3)}
Bの速度を$v$として
\begin{alignat*}{4}
  \text{運動量保存則}&:& Mv_1 &=& Mv &+& 2MV \\
  \text{エネルギー保存則}&:\quad& \frac{1}{2}M{v_1}^2 &=& \frac{1}{2}M{v}^2 &+& \frac{1}{2}\cdotp 2M{V}^2
\end{alignat*}
より
\begin{align*}
  V = \frac{2}{3}v_1
\end{align*}

\hang{III}
AとBを一体としてみたとき,その重心速度は$v_2$で一定である.
これが$u_1$より小さいことが条件である.
\begin{align*}
  v_2 > u_1 \qquad
  \therefore\ v_1 > 3u_1
\end{align*}


%%%%%%%%%%%%%%%%%%
