%!TEX root = *.tex
%%%%%%%%%%%%%%%%%%
% メモ
\begin{comment}

\end{comment}
% カウンタのリセット
\setcounter{eqNo}{0}
\setcounter{figure}{0}
% 解答
\noindent {\large【解答】\par}
2種類の経路$\text{S}_0\to\text{S}_1\to P$,$\text{S}_0\to\text{S}_2\to P$の長さの差は,
$\text{S}_1P$と$\text{S}_2P$の差に一致する.

\begin{align*}
  \text{S}_1P &= \sqrt{L^2+(2d+x)^2} = L\sqrt{1+\left(\frac{2d+x}{L}\right)^2}\\
  \text{S}_2P &= \sqrt{L^2+(2d-x)^2} = L\sqrt{1+\left(\frac{2d-x}{L}\right)^2}
\end{align*}
ここで,$2d+x\ll L$より与えられた近似式から
\begin{align*}
  \text{S}_1P-\text{S}_2P &\fallingdotseq \frac{4d}{L}x
\end{align*}
となるので,
\begin{empheq}[left=\empheqlbrace]{alignat*=4}
  明線条件 &\,:\,& \frac{4d}{L}x &= m\lambda &\quad& (m=0,\,\pm 1,\,\pm 2,\,\,\dots)  \\
  暗線条件 &\,:\,& \frac{4d}{L}x &= \left(m+\frac{1}{2}\right)\lambda &\quad& (m=0,\,\pm 1,\,\pm 2,\,\dots)
\end{empheq}

\noindent (2)\,
同じ$m$に対して,明線と暗線の位置はそれぞれ$x,\,x+a$と書き表せるので,
\begin{align*}
  &\begin{dcases}
    \frac{4d}{L}x = m\lambda \\
    \frac{4d}{L}(x+a) = (m+\frac{1}{2})\lambda
  \end{dcases} \\
  &\frac{4d}{L}a = \frac{\lambda}{2} \\
  &\therefore a = \frac{L\lambda}{8d}
\end{align*}

\noindent (3)\,
$\text{S}_0\text{S}_1=\sqrt{l^2+d^2}$,
$\text{S}_0\text{S}_2=\sqrt{l^2+3d^2}$となるので,$\frac{9d^2}{l^2}\ll 1$より
\begin{align*}
  \text{S}_0\text{S}_1-\text{S}_0\text{S}_2 &= \sqrt{l^2+d^2} - \sqrt{l^2+3d^2}\\
  &\fallingdotseq l\left(1+\frac{1}{2}\frac{9d^2}{l^2}\right) - l\left(1+\frac{1}{2}\frac{d^2}{l^2}\right) \\
  &= \frac{4d^2}{l}
\end{align*}
となり,$\text{S}_0\to\text{S}_1\to P$,$\text{S}_0\to\text{S}_2\to P$の経路差は
\begin{align*}
  (\text{S}_0\text{S}_1+\text{S}_1P)-(\text{S}_0\text{S}_2+\text{S}_2P) &\fallingdotseq \frac{4d}{L}x-\frac{4d^2}{l}
\end{align*}
よって,明線条件は
\begin{align*}
  \frac{4d}{L}x-\frac{4d^2}{l} = m\lambda \quad (m=0,\,\pm 1,\,\pm 2,\,\,\dots)
\end{align*}
以上より同じ$m$に対する明線の位置\x を考えると,明線の変位$\Delta x$は
\begin{align*}
  \frac{4d}{L}(x+\Delta x) - \frac{4d^2}{l} &= \frac{4d}{L}x \\
  \frac{4d}{L}\Delta x &= \frac{4d^2}{l} \\
  \Delta x &= \frac{Ld}{l}
\end{align*}
であり,$\Delta x$は正なので移動方向は右で,その距離が$\dfrac{Ld}{l}$である.

\noindent (4)\,
(3)と同様に暗線も$\Delta x$だけ移動する.
よって,明線と暗線の間隔は変わらないので,
$b=a=\frac{L\lambda}{8d}$.

\noindent (5)\,
屈折率\nn 中では光路長は経路長の\nn 倍となるので,
\begin{align*}
  \text{S}_0\to\text{S}_1\to P&: n\text{S}_0\text{S}_1+n\text{S}_1P \fallingdotseq nl\left(1+\frac{d^2}{2l^2}\right) + nL \left(1+\frac{1}{2}\frac{(2d+x)^2}{L^2}\right) \\
  \text{S}_0\to\text{S}_2\to P&: \text{S}_0\text{S}_2+n\text{S}_2P \fallingdotseq l\left(1+\frac{9d^2}{2l^2}\right) + nL \left(1+\frac{1}{2}\frac{(2d-x)^2}{L^2}\right)
\end{align*}

\noindent (6)\,
$x=0$としたときの光路長の差が0となるので,
\begin{align*}
  nl\left(1+\frac{d^2}{2l^2}\right) + nL \left(1+\frac{1}{2}\frac{4d^2}{L^2}\right) &= l\left(1+\frac{9d^2}{2l^2}\right) + nL \left(1+\frac{1}{2}\frac{4d^2}{L^2}\right)\\
  n\frac{2l^2+d^2}{2l} &= \frac{2l^2+9d^2}{2l} \\
  \therefore n &= \frac{2l^2+9d^2}{2l^2+d^2}
\end{align*}

\noindent (7)\,
(3),(4)より明線と暗線の間隔はついたてAとついたてBの間の光路長の差によらない.
よって,$\text{S}_1P$と$\text{S}_2P$の間の光路長の差についてのみ考えればよく,それは
\begin{align*}
  n\text{S}_1\P-n\text{S}_2\P = \frac{4nd}{L}x
\end{align*}
である.
(2),(4)と同様にして明線と暗線の間隔$c$は,
\begin{align*}
  \frac{4nd}{L}(x+c)-\frac{4nd}{L}x &= \frac{\lambda}{2} \\
  c &= \frac{L\lambda}{8nd} = \frac{L\lambda (2l^2+d^2)}{8d(2l^2+9d^2)}
\end{align*}
%%%%%%%%%%%%%%%%%%%%