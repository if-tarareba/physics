%!TEX root = *.tex
%%%%%%%%%%%%%%%%%%
% カウンタのリセット
\setcounter{figure}{0}
% 問題文

以下の\ajRoman{1}から\ajRoman{3}の文中にある\BrankNok{ア}から\BrankNok{シ}に適切な式や数値を入れよ.
ただし,必要ならば次の関係式を用いよ.
\begin{align*}
  \sin{A}+\sin{B} = 2\sin{\frac{A+B}{2}}\cos{\frac{A-B}{2}},\,
  \sin{A}-\sin{B} = 2\cos{\frac{A+B}{2}}\sin{\frac{A-B}{2}}
\end{align*}

\hang{\ajRoman{1}.}
放射能をもった一定量の原子核は,放射性崩壊により半減期$T$で時間とともに減少する.
\mbox{時間$t$}経過後に残っている原子核数$N$は,初期の原子核数を$N_0$とすると,$T$と$t$を用いて\BrankNok{ア}と表せる.
人工的に生成された原子番号53のヨウ素131の原子核は,$\beta$線を放出して崩壊($\beta$崩壊)し,16日後には初めの4分の1に減少するので,半減期$T$は\BrankNok{イ}日である.
このことから,48日後には,初めの\BrankNok{ウ}に減少すると推定できる.
なお,$\beta$崩壊したヨウ素131の原子核は,原子番号が\BrankNok{エ}のキセノン原子核となる.

\hang{\ajRoman{2}.}
X線は波長の短い電磁波で,真空中を光の速さ$c$と等しい速さで進む.
X線の波の性質を利用すると結晶の格子面間隔$d$を調べることができる.
いま,波長$\lambda$のX線を,格子面に対して角度$\theta$で入射する.
このとき,隣り合う格子面で反射されるX線の経路の長さに違いが生ずる.
そのため,隣り合う格子面で反射されたX線は,
波長$\lambda$,角度$\theta$および格子面間隔$d$の間に,
\nn を自然数として\BrankNok{オ}の条件が満たされるとき,
干渉して互いに強め合う.
このことから,波長$\lambda$が$0.154\,\text{nm}$のX線を用いて,$d$が$\BrankNok{カ}\,\text{nm}$以上の結晶の格子面間隔を調べることができる.\\
\hspace{.5zw}
一方,X線は粒子(光子)としての性質もある.
波長$\lambda$のX線光子は,真空中での運動量の大きさが\BrankNok{キ}で,
エネルギーの大きさが\BrankNok{ク}である粒子のようにふるまう.
ただし,プランク定数を$h$とした.

\hang{\ajRoman{3}.}
水面上に,三角形PRSの頂点P,R,Sの3点を考える.
いま,点Sと点Rを波源として,波長$\lambda$,振動数$f$,振幅$A$の波が減衰することなく,速さ$v$で広がっている.
点Pと点Sの間の距離は\x ,点Pと点Rの間の距離は\z で,
\x と\z は$\lambda$よりも十分に大きい($x \gg \lambda,\,z \gg \lambda$).
また,点Sと点Rの間の距離も$\lambda$より十分に大きい.
いま,時刻$t$で,点Sの波源による上下振動の変位が
$y_{\rm S}=A\sin{(\omega t+\frac{\pi}{3})}$,
点Rの波源による上下振動の変位が
$y_{\rm R}=A\sin{(\omega t+\frac{2\pi}{3})}$
と表せたとする.
ただし,$\omega =2\pi f$である.
このとき点Pには,時刻$t$で,点Sの波源から伝わる$y_1=\BrankNok{ケ}$の変位と,
点Rの波源から伝わる$y_2=\BrankNok{コ}$が現れる.
すなわち,点Pでは,点Sと点Rから伝わってきた波が重なる.
このとき点Pで観測される波の振幅は\BrankNok{サ}である.
このことから,点Pに現れる波の振幅は,波長$\lambda$,距離\x ,距離\z の間に,$m$を0以上の整数として,
\BrankNok{シ}の関係があるときに最大となる.


% メモ
\begin{comment}

\end{comment}


%%%%%%%%%%%%%%%%%%
