%!TEX root = *.tex
%%%%%%%%%%%%%%%%%%
% メモ
\begin{comment}

\end{comment}
% カウンタのリセット
% 解答
\hang{(3)}
静電誘導が発生して自由電子が移動しているときは,
電場が発生しているが,その電場は,
移動したことによって生じる電場によって打ち消される.
定常状態においては電場は0であり,全体が等電位である.

\hang{(5)}
絶縁体の構成粒子それぞれが極性を持ち,極板間の電場とは逆向きの電場が生じる.
そのため,電場が弱められる.
絶縁体を挿入する前後の電場をそれぞれ$E$,$E'$とすると,$E>E^\prime$である.

\hang{(6)}
極板間の間隔を$d$とすると,絶縁体の挿入前後の電位差$V=Ed$,$V'=E'd$は$V<V'$をみたす.

\hang{(7)}
コンデンサーに蓄えられた電荷を$Q$とすると,電気容量$C=\dfrac{Q}{V}$,$C'=\dfrac{Q}{V'}$は$C>C'$をみたす.

\hang{(8)}
時間$\varDelta t$\unit{s}の間に金属導線中のある断面を通過する自由電子の数は体積$A\cdotp v\varDelta t$\unit{$\text{m}^3$}の中に存在する自由電子の数$N\cdotp Av\varDelta t$である.
電気素量を用いて単位時間あたりに通過する電気量$I$を求めると
\begin{align*}
I&=\frac{e\cdotp NAv\varDelta t}{\varDelta t} = eNAv\unit{C/s=A}\\
\end{align*}

\hang{(9)}
導体の内部には一様な電場が存在していて,両端の間隔が$L$で電位差が$V$なので電場$E$は$E=\dfrac{V}{L}$\unit{V/m}となる.

\hang{(10)}
自由電子が電場から受ける力の大きさは$eE=\dfrac{eV}{L}$\unit{N}である.

\hang{(11)}
自由電子が電場から受ける力の向きは電場(電流)の向きとは逆なので,
電子の速度の向きと一致する.
また,電子は単位時間に$v$だけ移動するので,電場から受ける力のした仕事は$\dfrac{eV}{L}\times v = \dfrac{evV}{L}\unit{N$\cdotp$m/s=J/s=W}$.

\hang{(13)}
導線内に存在する自由電子の総数は$NAL$であるから,熱量$Q$は
\begin{align*}
  Q=NAL\times \frac{evV}{L} = eNAvV\unit{W}
\end{align*}

\hang{(14)}
$I=eNAv$より$Q=IV$.

\hang{(15)}
力のつりあいより
\begin{align*}
  kv &=\dfrac{eV}{L} \qquad \therefore
  v =\dfrac{eV}{kL}
\end{align*}

\hang{(16)}
オームの法則より
\begin{align*}
  R &= \dfrac{V}{eNAv} = \dfrac{V}{eNA}\dfrac{kL}{eV} = \dfrac{k}{e^2N}\dfrac{L}{A}
\end{align*}


%%%%%%%%%%%%%%%%%%
