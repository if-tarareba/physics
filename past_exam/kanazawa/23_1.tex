%!TEX root = *.tex
%%%%%%%%%%%%%%%%%%
% カウンタのリセット
\setcounter{figure}{0}
% 問題文

以下の文章が正しい記述となるように,
\BrankNo{(3)},\BrankNo{(5)},\BrankNo{(6)},\BrankNo{(7)}の\{\quad\}内の選択肢のいずれかを選びなさい.
また,それ以外の\BrankNo{\ }には適切な語句あるいは式を記入しなさい.
式には文中に与えられた文字と,必要ならば,電気素量$e\unit{C}$を用いなさい.

物質は,電気をよく通す導体と,ほとんど通さない不導体(絶縁体)に大別される.
銅やアルミなどの金属は導体の,ガラスやプラスチックなどは絶縁体の代表例である.
また,電気の通しやすさが導体と絶縁体の中間の物質があり,これを\BrankNo{(1)}という.
導体である金属には,金属を構成している個々の原始に属さずに金属内を自由に動き回る電子があり,これを自由電子という.
金属に帯電体を近づけると自由電子が静電気力によって移動する.
そのため,帯電体が正に帯電している場合を考えると,
帯電体に近い側の表面には負の電荷が現れ,遠い側には正の電荷が現れる.
この現象を\BrankNo{(2)}という.
このとき,導体は\,\fbox{\quad (3)\quad\{帯電体側が高電位・帯電体が低電位・全体が等電位\}\quad}\,となる.
一方,絶縁体では,電子はすべて構成粒子(原子,分子,イオン)に属し,自由電子がないため,電気を通しにくい.
絶縁体の電子は構成粒子から離れないが,帯電体を近づけると,静電気力によって構成粒子に属している電子の位置がずれる.
これを\BrankNo{(4)}という.
ここで絶縁体をコンデンサーに挿入することを考えよう.
まず,2枚の金属板からなる平行板コンデンサーの一方の極板に正電荷を与え,
この正電荷と大きさの等しい負電荷をもう一方の極板に与える.
その極板間を絶縁体で満たすと,極板上の電荷がつくる電場は\,\fbox{\quad (5)\quad\{強め・弱め\}\quad}\,られる.
そのため,極板間の電位差は\,\fbox{\quad (6)\quad\{大きく・小さく\}\quad}\,なり,
コンデンサーの電気容量は\,\fbox{\quad (7)\quad\{大きく・小さく\}\quad}\,なる.

次に,金属導線に電池をつなぎ電流を流すことを考える.
電流の大きさは単位時間あたりに導線の断面を通過する電気量の大きさである.
金属導線内の自由電子は電場により加速されるが,
熱運動している陽イオンなどから抵抗力を受ける.
やがて電場による力と陽イオンなどからの抵抗力が釣り合い,
自由電子は一定の速さ$v\unit{m/s}$で電場と逆向きに移動するとして電流の大きさを求めてみよう.
導線の断面積は$A\unit{$\text{m}^2$}$,長さは$L\unit{m}$である.
導線の単位体積中の自由電子の数を\mbox{$N$\unit{個/$\text{m}^3$}}とすると,
電流の大きさは\BrankNo{(8)}\unit{A}となる.
導線の両端には電圧$V\unit{V}$が加えられており,導線内部に大きさ\BrankNo{(9)}\unit{V/m}の一様な電場が生じている.
自由電子はそれぞれ電場から大きさ\BrankNo{(10)}\unit{N}の静電気力を受けるため,
電子1個が単位時間あたり電場からされる仕事は\BrankNo{(11)}\unit{W}となる.
この仕事は電子にはたらいている抵抗力により熱的なエネルギーに変換される.
この熱を\BrankNo{(12)}という.
単位時間あたりに長さ$L$の導線から発生する熱量は\BrankNo{(13)}\unit{W}となる.
\BrankNo{(8)}の電流の大きさを$I$とすると,\BrankNo{(13)}の熱量を$I$と電圧$V$を用いて\BrankNo{(14)}\unit{W}と表すことができる.

次に,自由電子が陽イオンなどから受けている抵抗力の大きさは速さに比例すると考えよう.
その比例係数を$k$\unit{N$\cdot$s/m}とする($k>0$).
また,以下の(15),(16)の解答に$v$と$I$を用いてはならない.
抵抗力と電場による力が釣り合っているときの自由電子の速さは\BrankNo{(15)}\unit{m/s}である.
すべての自由電子がこの速さで運動していると考え,電流の大きさを\BrankNo{(8)}から求めると,この関係式はオームの法則を表している.この結果から導線の電気抵抗は\BrankNo{(16)}\unit{$\Omega$}と求められ,比例定数$k$に比例することがわかる.

\vspace{\baselineskip}

\begin{comment}

\noindent
\textbf{【解答欄】}
\begingroup
\renewcommand{\arraystretch}{2}
\begin{table}[H]
  \centering
  \begin{tabular}{|p{.2\textwidth}|p{.2\textwidth}|p{.2\textwidth}|p{.2\textwidth}|}\hline
    (1)	& (2) & \multicolumn{2}{|l|}{(3)}\\\hline
    (4) &	(5)	& (6)	& (7) \\\hline
    (8)	& (9)	& (10) & (11) \\\hline
    \multicolumn{2}{|l|}{(12)} & (13) & (14) \\\hline
    \multicolumn{2}{|l|}{(15)} &	\multicolumn{2}{|l|}{(16)}\\\hline
  \end{tabular}
\end{table}
\endgroup

\end{comment}

%\begin{comment}
\noindent
\textbf{【解答】}
\begingroup
\renewcommand{\arraystretch}{2}
\begin{table}[H]
  \centering
  \begin{tabular}{|p{.2\textwidth}|p{.2\textwidth}|p{.2\textwidth}|p{.2\textwidth}|}\hline
    (1)	半導体& (2) 静電誘導& \multicolumn{2}{|l|}{(3) 全体が等電位}\\\hline
    (4) 誘電分極&	(5)	弱め& (6)	小さく& (7) 大きく\\\hline
    (8)	$eNAv$& (9)	$\dfrac{V}{L}$& (10) $\dfrac{eV}{L}$& (11) $\dfrac{evV}{L}$\\\hline
    \multicolumn{2}{|l|}{(12) ジュール熱} & (13) $eNAvV$& (14) $IV$\\\hline
    \multicolumn{2}{|l|}{(15) $\dfrac{eV}{kL}$} &	\multicolumn{2}{|l|}{(16) $\dfrac{k}{e^2n}\dfrac{L}{A}$}\\\hline
  \end{tabular}
\end{table}
\endgroup
%\end{comment}



% メモ
\begin{comment}

\end{comment}


%%%%%%%%%%%%%%%%%%
